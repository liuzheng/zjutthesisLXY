%%% Zhejiang University of Technology 
%%% !Mode\dots "TeX:UTF-8"
%%% author : liuzheng712@gmail.com(刘正)

LaTeX (formatted as LATEX, pronounced /ˈlɑːtɛx/, /ˈlɑːtɛk/, /ˈleɪtɛx/, or /ˈleɪtɛk/), is a document markup language and document preparation system for the TeX typesetting program. The term LaTeX refers only to the language in which documents are written, not to the editor application used to write those documents. In order to create a document in LaTeX, a .tex file must be created using some form of text editor. While most text editors can be used to create a LaTeX document, a number of editors have been created specifically for working with LaTeX.

LaTeX is widely used in academia.\cite{WrTex,Alexia} It is also used as the primary method of displaying formulas on Wikipedia. As a primary or intermediate format, e.g., translating DocBook and other XML-based formats to PDF, LaTeX is used because of the high quality of typesetting achievable by TeX. The typesetting system offers programmable desktop publishing features and extensive facilities for automating most aspects of typesetting and desktop publishing, including numbering and cross-referencing, tables and figures, page layout and bibliographies.

LaTeX is intended to provide a high-level language that accesses the power of TeX. LaTeX essentially comprises a collection of TeX macros and a program to process LaTeX documents. Because the TeX formatting commands are very low-level, it is usually much simpler for end-users to use LaTeX.

LaTeX was originally written in the early 1980s by Leslie Lamport at SRI International.[3] The current version is LaTeX2e (styled as LATEX2ε). LaTeX is free software and is distributed under the LaTeX Project Public License (LPPL)

\newcommand{\keywords}{\LaTeX}
