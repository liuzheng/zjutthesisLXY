%%% Zhejiang University of Technology 
%%% !Mode\dots "TeX:UTF-8"
%%% author: liuzheng712@gmail.com(刘正)


\chapter{符号}
\section{特殊符号}
\song\wuhao
\begin{table}[!hbp]
\begin{tabular}{|*{11}{c|}}\hline
输出字符 & \# & \$ & \% & \{ & \} & \~{} & \_{} & \^{} & \textbackslash & \& \\\hline
输入命令 &\verb|\#| &\verb|\$| &\verb|\%| &\verb|\{|
         &\verb|\| &\verb|\~{| &\verb|\_{| &\verb|\^{|
         &\verb|\textbackslash| &\verb|\&|\\\hline
\end{tabular}
\end{table}

\bigskip
\begin{table}[!hbp]
\begin{tabular}{|*{5}{c|}}\hline
输出字符 & $|$ & $>$ & $<$ & $\backslash$  \\\hline
输入命令 &\verb+$|$+ &\verb|$>$| &\verb|$<$| &\verb|$\backslash$|\\\hline
\end{tabular}
\end{table}

\bigskip
\begin{table}[!hbp]
\begin{tabular}{|*{7}{c|}}\hline
输出字符 & \S & \P & \dag & \ddag & \copyright & \pounds \\\hline
输入命令 &\verb|\S| &\verb|\P| &\verb|\dag| &\verb|\ddag|
         &\verb|\copyright| &\verb|\pounds| \\\hline
\end{tabular}
\end{table}

\bigskip
\begin{table}[!hbp]
\begin{tabular}{|*{5}{c|}}\hline
输出字符 & \TeX & \LaTeX & \LaTeXe & \XeLaTeX \\\hline
输入命令 &\verb|\TeX| &\verb|\LaTeX| &\verb|\LaTeXe| &\verb|\XeLaTeXe|\\\hline
\end{tabular}
\end{table}

\bigskip
\begin{table}[!hbp]
\begin{tabular}{|*{6}{c|}}\hline
输出字符 & `~(左单引号) & '~(右单引号) & ``~(左双引号) & "~(右双引号) & `\,``\\\hline
输入命令 &\`{}~(倒引号) &\verb|'|~(单引号) &\`{}\`{}~(两个倒引号)
         &\verb|"|~(双引号) &\`{}\verb|\,|\`{}\`{}\\\hline
\end{tabular}
\end{table}

\bigskip
\begin{table}[!hbp]
\begin{tabular}{|*{5}{c|}}\hline
输出字符  & -~(连字符) & --~(数字范围) & ---~(破折号) & $-$~(减号)\\\hline
输入命令  &\verb|-| &\verb|--| &\verb|---| &\verb|$-$| \\\hline
\end{tabular}
\end{table}

\bigskip
\begin{table}[!hbp]\begin{tabular}{|*{8}{c|}}\hline
输出字符  & {\oe} & {\OE} & {\ae} & {\AE} & {\aa} & {\AA} & {\o} \\\hline
输入命令  &\verb|{\oe| &\verb|{\OE| &\verb|{\ae| &\verb|{\AE|
          &\verb|{\aa| &\verb|{\AA| &\verb|{\o|\\\hline\hline
输出字符  & {\O} & {\l} & {\L} & {\ss} & {\SS} & !` & ?` \\\hline
输入命令  &\verb|{\O| &\verb|{\l| &\verb|{\L| &\verb|{\ss|
          &\verb|{\SS| &\verb|!|\`{} &\verb|?|\`{}\\\hline
\end{tabular}\end{table}

\section{数学符号表}
下面的表格中将给出在数学模式中常用的所有符号。使用表 3.12–3.167
所列出的符号,必须事先安装 AMS 数学字库并且在文档的导言区加载宏包: amssymb。如果你的系统中没有安装 AMS 宏包和数学字库,可去下述

地址下载:

CTAN:/tex-archive/macros/latex/required/amslatex

\begin{table}[H]
% 数学模式重音符
\centering
\begin{tabular}{*{8}{c}}
$\hat{a}$ & \verb|{\hat{a}| & $\check{a}$ & \verb|\check{a}| & $\tilde{a}$ & \verb|\tilde{a}| & $\acute{a}$ & \verb|\acute{a}| \\
$\grave{a}$ & \verb|\grave{a}| & $\dot{a}$ & \verb|\dot{a}| & $\ddot{a}$ & \verb|\ddot{a}| & $\breve{a}$ & \verb|\breve{a}| \\
$\bar{a}$ & \verb|\bar{a}| & $\vec{a}$ & \verb|\vec{a}| & $\widehat{A}$ & \verb|\widehat{A}| & $\widetilde{A}$ & \verb|\widetilde{A}| \\
\end{tabular}
\end{table}

\begin{table}[H]
% 小写希腊字母
\centering
\begin{tabular}{*{8}{c}}
$\alpha$ & \verb|\alpha| & $\theta$ & \verb|\theta| & $o$ & \verb|o| & $\upsilon $& \verb|\upsilon| \\
$\beta$ & \verb|\beta| & $\vartheta$ & \verb|\vartheta| & $\pi$ & \verb|\pi| & $\phi$ & \verb|\phi| \\
$\gamma$ & \verb|\gamma| & $\iota$ & \verb|\iota| & $\varpi $& \verb|\varpi| & $\varphi $& \verb|\varphi| \\
$\delta$ & \verb|\delta| & $\kappa$ & \verb|\kappa| & $\rho $& \verb|\rho| & $\chi $& \verb|\chi| \\
$\epsilon$ & \verb|\epsilon| & $\lambda$ & \verb|\lambda| & $\varrho$ & \verb|\varrho| &$ \psi $& \verb|\psi| \\
$\varepsilon$ & \verb|\varepsilon| & $\mu$ & \verb|\mu| & $\sigma$ & \verb|\sigma| & $\omega$ & \verb|\omega| \\
$\zeta$ & \verb|\zeta| & $\nu$ & \verb|\nu| & $\varsigma$ & \verb|\varsigma| &  &  \\
$\eta$ & \verb|\eta| & $\xi$ & \verb|\xi| & $\tau$ & \verb|\tau| &  &  \\

\end{tabular}
\end{table}

\begin{table}[H]
% 小写希腊字母
\centering
\begin{tabular}{*{8}{c}}
$\Gamma $ & \verb|\Gamma| &  $ \Lambda$ & \verb|\Lambda| & $\Sigma $ & \verb|\Sigma| &  $\Psi $ & \verb|\Psi| \\
$\Delta $ & \verb|\Delta| &  $\Xi $ & \verb|\Xi| &  $\Upsilon $ & \verb|\Upsilon| &  $ \Omega$ & \verb|\Omega| \\
$\Theta $ & \verb|\Theta| &  $\Pi $ & \verb|\Pi| &  $ \Phi$ & \verb|\Phi| &  $ $ &  \\
\end{tabular}
\end{table}



\subsection{二元运算符}

\begin{table}[H]
% 二元运算符
\centering
\begin{tabular}{*{6}{c}}
$+ $&\verb|+| &$- $&\verb|-| && \\
$\pm $&\verb|\pm| &$\mp $&\verb|\mp| &$\triangleleft $&\verb|\triangleleft| \\
$\cdot $&\verb|\cdot| &$\div $&\verb|\div| &$\triangleright $&\verb|\triangleright| \\
$\times $&\verb|\times| &$\setminus $&\verb|\setminus| &$\star $&\verb|\star| \\
$\cup $&\verb|\cup| &$\cap $&\verb|\cap| &$\ast $&\verb|\ast| \\
$\sqcup $&\verb|\sqcup| &$\sqcap $&\verb|\sqcap| &$\circ $&\verb|\circ| \\
$\vee $&\verb|\vee,\lor| &$\wedge $&\verb|\wedge,\land| &$\bullet $&\verb|\bullet| \\
$\oplus $&\verb|\oplus| &$\ominus $&\verb|\ominus| &$\diamond $&\verb|\diamond| \\
$\odot $&\verb|\odot| &$\oslash$&\verb|\oslash| &$\uplus $&\verb|\uplus| \\
$\otimes $&\verb|\otimes| &$\bigcirc$&\verb|\bigcirc| &$\amalg $&\verb|\amalg| \\
$\bigtriangleup $&\verb|\bigtriangleup| &$\bigtriangledown $&\verb|\bigtriangledown| &$\dagger $&\verb|\dagger| \\
$\lhd$&\verb|\lhd|\hyperlink{latexsym}{\footnotemark[1]} &$\rhd $&\verb|\rhd|\hyperlink{latexsym}{\footnotemark[1]} &$\ddagger $&\verb|\ddagger| \\
$\unlhd $&\verb|\unlhd|\hyperlink{latexsym}{\footnotemark[1]} &$\unrhd $&\verb|\unrhd|\hyperlink{latexsym}{\footnotemark[1]} &$\wr $&\verb|\wr|
\end{tabular}
\end{table}
\footnotetext[1]{\song \hypertarget{latexsym}{使用宏包 latexsym 来得到这个符号}}

\begin{table}[H]
% 大尺寸运算符
\centering
\begin{tabular}{*{8}{c}}
$\sum $ & \verb|\sum| &$\bigcup $ & \verb|\bigcup| &$\bigvee $ & \verb|\bigvee| &$ \bigoplus$ & \verb|\bigoplus| \\
$\prod $ & \verb|\prod| &$\bigcap $ & \verb|\bigcap| &$\bigwedge $ & \verb|\bigwedge| &$ \bigotimes$ & \verb|\bigotimes| \\
$\coprod $ & \verb|\coprod| &$\bigsqcup $ & \verb|\bigsqcup| & &  &$\bigodot $ & \verb|\bigodot| \\
$ \int$ & \verb|\int| &$\oint $ & \verb|\oint| &  &  &$ \biguplus$ & \verb|\biguplus| 
\end{tabular}
\end{table}


\begin{table}[H]
% 箭头
\centering
\begin{tabular}{*{6}{c}}
$\leftarrow $ & \verb|\leftarrow or \gets| &$\longleftarrow $ & \verb|\longleftarrow| &$\uparrow $ & \verb|\uparrow| \\
$\rightarrow $ & \verb|\rightarrow or \to| &$ \longrightarrow$ & \verb|\longrightarrow| &$ \downarrow$ & \verb|\downarrow| \\
$\leftrightarrow $ & \verb|\leftrightarrow| &$\longleftrightarrow $ & \verb|\longleftrightarrow| &$\updownarrow $ & \verb|\updownarrow| \\
$ \Leftarrow$ & \verb|\Leftarrow| &$\Longleftarrow $ & \verb|\Longleftarrow| &$\Uparrow $ & \verb|\Uparrow| \\
$\Rightarrow $ & \verb|\Rightarrow| &$\Longrightarrow $ & \verb|\Longrightarrow| &$ \Downarrow$ & \verb|\Downarrow| \\
$\Leftrightarrow $ & \verb|\Leftrightarrow| &$ \Longleftrightarrow$ & \verb|\Longleftrightarrow| &$ \Updownarrow$ & \verb|\Updownarrow| \\
$\mapsto $ & \verb|\mapsto| &$\longmapsto  $ & \verb|\longmapsto | &$\nearrow $ & \verb|\nearrow| \\
$\hookleftarrow $ & \verb|\hookleftarrow| &$\hookrightarrow $ & \verb|\hookrightarrow| &$ \searrow$ & \verb|\searrow| \\
$ \leftharpoonup$ & \verb|\leftharpoonup| &$\rightharpoonup $ & \verb|\rightharpoonup| &$\swarrow $ & \verb|\swarrow| \\
$\leftharpoondown$ & \verb|\leftharpoondown| &$ \rightharpoondown$ & \verb|\rightharpoondown| &$\nwarrow $ & \verb|\nwarrow| \\
$\rightleftharpoons $ & \verb|\rightleftharpoons| &$ \iff$ & \verb|\iff (bigger spaces)| &$\leadsto $ & \verb|\leadsto|\hyperlink{latexsym}{\footnotemark[1]} \\

\end{tabular}
\end{table}

\begin{table}[H]
% 定界符
\centering
\begin{tabular}{*{8}{c}}
$( $ & \verb|(| &$ )$ & \verb|)| &$\uparrow $ & \verb|\uparrow| &$\Uparrow $ & \verb|\Uparrow|\\
$ [$ & \verb|[ or \lbrack| &$ ]$ & \verb|] or \rbrack| &$ \downarrow$ & \verb|\downarrow| &$ \Downarrow$ & \verb|\Downarrow| \\
$\{ $ & \verb|\{ or \lbrace| &$\}$ & \verb|\} or \rbrace| &$\updownarrow $ & \verb|\updownarrow| &$ \Updownarrow$ & \verb|\Updownarrow| \\
$\langle $ & \verb|\langle| &$\rangle $ & \verb|\rangle| &$| $ & \verb|| or \vert| &$\| $ & \verb|\| or \Vert| \\
$\lfloor $ & \verb|\lfloor| &$\rfloor $ & \verb|\rfloor| &$\lceil $ & \verb|\lceil| &$\rceil $ & \verb|\rceil| \\
$/ $ & \verb|/| &$\backslash $ & \verb|\backslash| &$ $ & \verb|.(dual. empty)| & & 
\end{tabular}
\end{table}

\begin{table}[H]
% 大尺寸定界符
\centering
\begin{tabular}{*{8}{c}}
$\lgroup $ & \verb|\lgroup| &$\rgroup $ & \verb|\rgroup| &$\lmoustache $ & \verb|\lmoustache| &$ \rmoustache$ & \verb|\rmoustache| \\
$\arrowvert $ & \verb|\arrowvert| &$\Arrowvert $ & \verb|\Arrowvert| &$\bracevert $ & \verb|\bracevert| & &
\end{tabular}
\end{table}
%$ $ & \verb|| &$ $ & \verb|| &$ $ & \verb|| \\
%$ $ & \verb|| &$ $ & \verb|| &$ $ & \verb|| \\
%$ $ & \verb|| &$ $ & \verb|| &$ $ & \verb|| \\
%$ $ & \verb|| &$ $ & \verb|| &$ $ & \verb|| \\
%$ $ & \verb|| &$ $ & \verb|| &$ $ & \verb|| \\
%$ $ & \verb|| &$ $ & \verb|| &$ $ & \verb|| \\
%$ $ & \verb|| &$ $ & \verb|| &$ $ & \verb|| \\
%$ $ & \verb|| &$ $ & \verb|| &$ $ & \verb|| \\
%$ $ & \verb|| &$ $ & \verb|| &$ $ & \verb|| \\\end{tabular}



\section{英语音标符号及其\LaTeX表达式}
\song\wuhao
下表中的音标大部分是通过tipa 宏包实现的(其作者为日本东京大学的Rei Fukui),在输入音标前首先要包含tipa 包,即$\backslash$usepackage\{tipa\}。

如果是在\XeLaTeX中使用tipa 包,则请在导言区加入$\backslash$setmainfont\{Times New Roman\}或其他
能够将西文字体设定为Times New Roman的指令,否则某些音标符号将无法显示

\begin{table}[H]
\song\wuhao
\centering
\begin{tabular}{|c|c|c|c|c|c|}
\hline
音标符号 & \LaTeX表达式 &音标符号 & \LaTeX表达式 &音标符号 & \LaTeX表达式 \\
\hline
\times\i & $\backslash$i & \times\textdyoghlig & $\backslash$textdyoghlig & \times\textscripta &$\backslash$textscripta\\
\hline
\times\ae & $\backslash$ae & \times\textturnv & $\backslash$textturnv & \times\dh & $\backslash$dh\\
\hline
\times\j & $\backslash$j & \times\textschwa & $\backslash$textschwa & \times\textepsilon & $\backslash$textepsilon \\
\hline
\times\textdzlig & $\backslash$textdzlig & \times\textscriptg & $\backslash$textscriptg & \times\textopeno & $\backslash$textopeno \\
\hline
\times\textesh & $\backslash$textesh & \times\texttheta & $\backslash$texttheta & \times\textyogh & $\backslash$textyogh\\
\hline
\times\textteshlig & $\backslash$textteshlig & \times\textupsilon & $\backslash$textupsilon & \times\ng & $\backslash$ng \\
\hline
\times\textprimstress\song(重音) & $\backslash$textprimstress & \times\textsecstress\song(次重音) & $\backslash$textsecstress & \times\textlengthmark\song(长音) & $\backslash$textlengthmark \\
\hline
\end{tabular}
\end{table}

上表中未列出的音标大多都可以直接由键盘输入而不需要任何 \LaTeX指令,比如l、e、k、b
等等;或者可由以上音标的\LaTeX表达式组合拼接得到,比如{\times\textopeno\textlengthmark}、{\times\textepsilon\textschwa}等等。

另外,在英文文章中也常常见到一些上面加两个点的字母,例如{\times Schr\"odinger}(薛定谔)、
{\times na\"ive}(幼稚的),这也可以用\LaTeX实现:{\times\"o}( $\backslash$"o)、{\times\"i}($\backslash$"i)、{\times\"u}($\backslash$"u)等,需要注意的是反斜
杠$\backslash$后面的一定是\fbox{英文标点中的双引号}。

以上仅仅是英语中普遍使用的音标符号,其它IPA(国际音标符号)符号的\LaTeX表达式请参阅tipa 宏包的说明文档(打开Windows的命令提示符,输入{\bfseries\times texdoc tipa},在打开的窗口中选
\textbf{\times tipaman.pdf})的附录部分

\begin{table}[H]
\centering
\begin{tabular}{|*{8}{c|}}\hline
音标 & \LaTeX表达式 &音标 & \LaTeX表达式 &音标 & \LaTeX表达式&音标 & \LaTeX表达式 \\
\hline
\`o & \verb|\`o| &
\'o & \verb|\'o| &
\^o & \verb|\^o| &
\"o & \verb|\"o| \\\hline
\~o & \verb|\~o| &
\=o & \verb|\=o| &
\.o & \verb|\.o| &
\u{o} & \verb|\u{o}| \\\hline
\v{o} & \verb|\v{o}| &
\H{o} & \verb|\H{o}| &
\r{o} & \verb|\r{o}| &
\t{oo} & \verb|\t{oo}| \\\hline
\b{o} & \verb|\b{o}| &
\c{o} & \verb|\c{o}| &
\d{o} & \verb|\d{o}| &
& \\\hline
\end{tabular}

\end{table}