%%% Zhejiang University of Technology 
%%% !Mode\dots "TeX:UTF-8"
%%% author : liuzheng712@gmail.com(刘正)

\XeTeXlinebreaklocale ``zh''
\XeTeXlinebreakskip = 0pt plus 1pt minus 0.1pt
\usepackage{fontspec,xunicode,xltxtra}
\usepackage{titlesec}			% 自定义章节标题包
\usepackage{titletoc}			% 自定义章节目录包
\usepackage{hyperref}			% 超链接
%\usepackage[colorlinks,linkcolor=black,anchorcolor=black,citecolor=black]{hyperref}	% 超链接如果不想要红框的话,打印时是不会打出框的
\usepackage{color}              % 彩色
\usepackage{xcolor}             % 彩色
\usepackage{listings}           % 代码风格
\usepackage{courier}
\usepackage[boldfont]{xeCJK}				% CJK字体宏包
\usepackage{setup/xCJKnumb}
%\usepackage{times}
\setmainfont{Times New Roman}	% 设定默认字体为times
%\setCJKmainfont{Adobe Song Std} 	% 设定中文为宋体
%\usepackage{CJKutf8}                % 中文
\usepackage{graphicx}           % 图片
\usepackage{subfigure}
\usepackage{caption}
\usepackage{geometry}           % 页面设置
\geometry{left=2.5cm,right=2.5cm,bottom=2.5cm,top=3cm}
\usepackage{indentfirst}        % 首行缩进
\setlength{\parindent}{2em}     % 首行缩进2字符
\usepackage{amstext}            % 使公式中使用text标签现实中文
\usepackage{fancyhdr}           % 页眉页脚
\usepackage{ulem}				% 下划线
\usepackage{pdfpages}           % 用于嵌入pdf
\usepackage{multirow}           % 使用多栏宏包
\usepackage{longtable}
\usepackage{amsmath}
\usepackage{amssymb}
%\usepackage{tipa}				% tipa音标宏包(发现有bug什么,而且不用也可以)
\usepackage[timestamp,first]{draftcopy}	% 水印宏包
\usepackage{float}				% 表格固定
\usepackage{latexsym}			% 获得特殊数学二元运算符
%\DeclareCaptionFont{white}{\color{white}}
%`\DeclareCaptionFormat{listing}{\colorbox[cmyk]{0.43, 0.35, 0.35,0.01}{\parbox{\textwidth}{#1#2#3}}}
%\captionsetup[lstlisting]{format=listing,justification=raggedright,labelfont=white,textfont=white, singlelinecheck=false, margin=0pt, font={sf,bf,normalsize}}



%\RequirePackage[subfigure]{ccaption}
